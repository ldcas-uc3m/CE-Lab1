\section{Methodology}

\subsection{Dictionaries}
I used 5 dictionaries for the attack:
\begin{enumerate}
    \item \textbf{Top 500 worst passwords:} A list of the 500 most common passwords, this will cover the general cases for passwords.
    \item \textbf{hotmail.txt:} A list of ~9000 (allegded) leaked passwords from Hotmail, also covering for general cases.
    \item \textbf{200 spanish names:} A list of the most common spanish names, as we assume the passwords are (mainly) from spanish people.
    \item \textbf{lab-inf-uc3m:} Dictionary generated from the webpage of the university's Computer Lab's website, using Cewl.\\
    This should help with some specific university and computer terminology.
    \item \textbf{rockyou.txt:} The de-facto standard dictionary for attacks.
\end{enumerate}

\subsection{Rules}
I implemented some of my own rules to the previously mentioned dictionaries in order to make them more complete:
\begin{enumerate}
    \item \textbf{newL33t:} My own (arguably naive) implementation of l33t sp34k, done by replacing the following letters (both upper and lowercase):
    \begin{enumerate}
        \item i $\rightarrow$ 1
        \item i $\rightarrow$ !
        \item o $\rightarrow$ 0
        \item a $\rightarrow$ 4
        \item l $\rightarrow$ 7
        \item s $\rightarrow$ \$
    \end{enumerate}
    \item \textbf{Numbers:} Appending one, two, or three numbers to the end or the beginning of the word.
    \item \textbf{Years:} Assuming some people would add their birthyear at the end of their password, we append some possible years (1930-2029).
    \item \textbf{Punctuation:} We append a couple of punctuation characters to the beginning and end of the words.
    \item \textbf{UC3M:} The university initials. We try (both upper and lowercase) at the beginning and the end.
    \item \textbf{Cases:} A few upper and lowecase toggles (all characters), as well as a general one.
    \item \textbf{sArCaSm:} Toggling even and odd upper and lower cases.
    \item \textbf{Tildes:} Replacing spanish accentuated vowels with their unaccentuated counterparts.\\
    For some reason, I wasn't able to implement it with John's Rule syntax.
\end{enumerate}


\subsection{Incremental}
As per teacher suggestion, I tried to test for random small passwords, by using the incremental method.